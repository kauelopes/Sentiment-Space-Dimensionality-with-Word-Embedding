
Corpos flutuantes são elementos não textuais, como figuras e tabelas, que complementam as informações do texto. Neste capítulo são expostos breves exemplos dos corpos flutuantes disponíveis na classe \textit{icmc}.

Na \autoref{secao:figuras} é mostrado como inserir figuras, a \autoref{secao:tabelas_e_quadros} explica como incluir tabelas e quadros, a \autoref{secao:algoritmos_e_codigos} demostra como trabalhar com algoritmos e códigos-fonte e a \autoref{secao:outros-ambientes} explica como definir outros ambientes para serem utilizados, como para gráficos e diagramas.

\section{Figuras}
\label{secao:figuras}

A inserção de figuras é realizada normalmente através do comando \comando{begin\{figure\}}. Na \autoref{figura:logomarca_usp} é exibida a logomarca da USP com o pacote \textit{graphicx}. Já a \autoref{figura:exemplo_grafo} mostra um exemplo de grafo com o pacote \textit{xy}. De acordo com as normas ABNT a lista de figuras é um elemento opcional do documento, para incluí-la é preciso inserir o comando \comando{incluidelistafiguras} antes do início do documento.

Observe que, segundo a \citeonline[seções 4.2.1.10 e 5.8]{NBR14724:2011}, as
ilustrações devem sempre ter numeração contínua e única em todo o documento. Além disso, deve ser incorporado ao corpo flutuante do tipo figura, além da legenda, a fonte de onde esta foi extraída. Se a figura foi confeccionada pelo próprio autor, deve se colocar "Elaborada pelo autor".

\begin{citacao}
Qualquer que seja o tipo de ilustração, sua identificação aparece na parte
superior, precedida da palavra designativa (desenho, esquema, fluxograma,
fotografia, gráfico, mapa, organograma, planta, quadro, retrato, figura,
imagem, entre outros), seguida de seu número de ordem de ocorrência no texto,
em algarismos arábicos, travessão e do respectivo título. Após a ilustração, na
parte inferior, indicar a fonte consultada (elemento obrigatório, mesmo que
seja produção do próprio autor), legenda, notas e outras informações
necessárias à sua compreensão (se houver). A ilustração deve ser citada no
texto e inserida o mais próximo possível do trecho a que se
refere. \cite[seções 5.8]{NBR14724:2011}
\end{citacao}

\begin{figure}[htb]
 \caption{Logomarca da USP}
 \label{figura:logomarca_usp}
 \centering
 \includegraphics[scale=0.3]{images/usp-logo}
 \fdireta{usp:logo}
\end{figure}


\begin{figure}[htb]
\caption{Exemplo de grafo}
\label{figura:exemplo_grafo}
\centering
\begin{scriptsize}
$$
\xymatrix@R20pt@C10pt{
 & & & & vr \ar[dlll] \ar[dl] \ar[d] \ar[dr] \ar[drr] \ar[drrr] & & & \\
 & (a_3, b_2, c_1) \ar[d]^{\varphi_2} \ar[dl]_{\varphi_1} & & (a_3, b_2, c_2) \ar[d]^{\varphi_2} \ar[dl]_{\varphi_1} & (a_1, b_1, c_1) & (a_1, b_1, c_2) & (a_1, b_2, c_1) & (a_1, b_2, c_2) \\
 (a_2, b_2, c_1) \ar[dr]_{\varphi_3} & (a_3, b_1, c_1) \ar[d]^{\varphi_1} & (a_2, b_2, c_2) \ar[dr]_{\varphi_3} & (a_3, b_1, c_2) \ar[d]^{\varphi_1} & & & & \\
& (a_2, b_1, c_1)  & & (a_2, b_1, c_2) & & & & \\
}
$$
\end{scriptsize}
\fautor
\end{figure}

A classe \textit{icmc} traz algum comando que auxiliam na inserção da legenda, para utilizá-los basta substituir o \comando{fonte\{\}} por um dos seguintes comando conforme necessário:

\begin{description}

 \item[\comando{fautor}] Insere o texto \aspas{Elaborada pelo autor} como fonte da figura;

 \item[\comando{fadaptada[INF]\{REF\}}] Insere um texto informando que a figura foi adaptada de alguma referência bibliográfica (REF). INF refere-se ao local específico de onde a imagem foi extraída, como por exemplo o número da página. Além disso, INF é um parâmetro opcional e pode receber qualquer cadeia de texto;

 \item[\comando{fdireta[INF]\{REF\}}] Insere um texto informando que a figura próvem diretamente de alguma referência bibliográfica (REF). INF refere-se ao local específico de onde a imagem foi extraída, como por exemplo o número da página. Além disso, INF é um parâmetro opcional e pode receber qualquer cadeia de texto;
 
 \item[\comando{fdadospesquisa}] Insere o texto \aspas{Dados da pesquisa.} como fonte da figura;
 
\end{description}



%\section{Tabelas e Quadros}
\label{secao:tabelas_e_quadros}

A inserção de tabelas e quadros é feita de forma semelhante a inserção de figuras, porém são utilizados os ambientes \textit{table} e \textit{quadro}. A principal diferença entre tabelas e quadros, de acordo com \citeonline{silveira:2006:manual_tcc}, é que as tabelas são destinadas para informações numéricas e os quadros são mais adequados para informações textuais. Em geral, as tabelas devem estar padronizadas conforme o padrão do
\citeonline{ibge1993} requerido pelas normas da ABNT para documentos técnicos e
acadêmicos.

Como exemplos foram inseridas a \autoref{tabela:lista_produtos} que exibe uma de lista de produtos (construída em \LaTeX) e a Tabela \autoref{tabela:populacao_america_sul} que mostra a população dos países da América do Sul (construída segundo o padrão do IBGE). Foi inserido também o \autoref{quadro:editores_texto_livres} com alguns editores que podem ser usados para se trabalhar com \LaTeX para demonstrar a inserção de quadros.

 A lista de tabelas também é um elemento opcional que pode ser incluída com o comando \comando{incluidelistatabelas} antes do início do documento. O mesmo acontece com a lista de quadros que pode ser incluída com o comando \comando{incluidelistaquadros}.

\begin{table}[htb]
\centering
\caption{Lista de produtos}
\label{tabela:lista_produtos}
\begin{tabularx}{\textwidth}{X|l|r|r|r} \hline
Produto      & Unidade & Preço (R\$) & Quantidade & Total (R\$) \\ \hline
Arroz        & Kg      & 2,00        & 550        & 1.100,00    \\
Óleo de Soja & L       & 2,50        & 500        & 750,00      \\
Açucar       & Kg      & 3,00        & 100        & 300,00      \\ \hline
\end{tabularx}
\fdadospesquisa
\end{table}

\begin{table}[htb]
\IBGEtab{%
  \caption{População dos países da América do Sul} \label{tabela:populacao_america_sul}
}{%
\begin{tabular}{r|p{3cm}|r}        
\toprule
Código  & País            & População   \\ \midrule \midrule
1       & Brasil          & ~~~~~~191.480.630 \\ \midrule 
2       & Argentina       &  39.934.100 \\ \midrule 
3       & Colômbia        &  46.741.100 \\ \midrule 
4       & Paraguai        &   9.694.200 \\ \midrule 
5       & Uruguai         &   3.350.500 \\ \midrule 
6       & Peru            &  28.221.500 \\ \midrule 
7       & Equador         &  13.481.200 \\ \midrule 
8       & Bolívia         &   9.694.200 \\ \midrule 
9       & Venezuela       &  28.121.700 \\ \midrule 
10      & Chile           &  16.803.000 \\ \bottomrule
\end{tabular}
}{%
  \fdireta{wikipedia:2011:america_sul}%
  \nota{Esta é uma nota, que diz que os dados são baseados na
  regressão linear.}%
  \nota[Anotações]{Uma anotação adicional, que pode ser seguida de várias
  outras, porém são opcionais.}%
  }
\end{table}

\begin{quadro}[htb]
\caption{Editores de Texto Livres}
\label{quadro:editores_texto_livres}
\centering
\begin{tabular}{|l|l|r|}        \hline
Editor     & Multiplataforma & Específico para Latex \\ \hline
Kwriter    & Sim             & Não                   \\
Texmaker   & Sim             & Sim                   \\
Kile       & Sim             & Sim                   \\
Geany      & Sim             & Não                   \\ \hline
\end{tabular}
\end{quadro}

\section{Algoritmos e Códigos}
\label{secao:algoritmos_e_codigos}

Além dos corpos flutuantes convencionais para inserir figuras (\comando{begin\{figure\}}) e tabelas (\comando{begin\{table\}}), a classe \textit{icmc} possui mais dois tipos de corpos flutuantes um para algoritmos (\comando{begin\{algoritmo\}}) e outro para códigos-fonte (\comando{begin\{codigo\}}). A utilização de um ou de outro fica a critério do usuário. Como exemplo temos o \autoref{algoritmo:mdc1} que calcula o máximo divisor comum entre dois números e os Códigos-fonte \ref{codigo:notas_alunos} e \ref{codigo:metodo_leitura} que são uma consulta na \sigla{SQL}{\textit{Structured Query Language}} e uma sobrotina em \textit{Java}.

%\begin{algoritmo}[htb]
\begin{algoritmo}
%\begin{algorithmic}[1]
\caption{Algoritmo para cálculo de máximo divisor comum MDC($n_1$,$n_2$)}
\label{algoritmo:mdc1}

 \KwIn{Dois números inteiros ($n_1, n_2$)}
 \If(\tcp*[f]{Garante que o maior número seja $n_1$}){$n_2 > n_1$}
   {troca valores de $n_1$ e $n_2$}
 \Repeat{$r > 0$}{
    $r \leftarrow$ resto da divisão de $n_1$ por $n_2$
    $n_1 \leftarrow n_2$
    $n_2 \leftarrow r$
 }
 \Return $n_1$
%\end{algorithmic}
\end{algoritmo}
%\end{algoritmo}

%\begin{codigo}[htb]
%\caption{Consulta SQL}
%\label{codigo:notas_alunos}
%\hrule
\begin{codigo}[caption = {Consulta SQL}, label={codigo:notas_alunos},language=SQL, breaklines=true]
SELECT a.nome_aluno AS aluno,
       d.nome_disciplina AS disciplina,
       m.nota AS nota
FROM aluno AS a,
     disciplina AS d,
     matriculado AS m
WHERE a.id_aluno = m.id_aluno
  AND d.id_disciplina = m.id_disciplina
ORDER BY a.nome_aluno, d.nome_disciplina;
\end{codigo}
%\end{codigo}

%\begin{codigo}[htb]
%\caption{Subrotina para obter uma entrada do usuário}
%\label{codigo:metodo_leitura}
%\hrule
\begin{codigo}[caption={Subrotina para obter uma entrada do usuário}, label={codigo:metodo_leitura}, language=Java, breaklines=true]
public static String Leitura(){
    BufferedReader reader = new BufferedReader(new InputStreamReader(System.in));
    try {
        return reader.readLine(); // Lê uma linha pelo teclado
    } catch (IOException e) {
        e.printStackTrace();
        return "";
    }
}
\end{codigo}
%\end{codigo}

Existem diversos outros pacotes disponíveis para escrever algoritmos e códigos. Nos exemplos anteriormente foram utilizados o pacote \textit{algorithm} para definição do ambiente algoritmo e \textit{listings} para a definição do ambiente de código-fonte. O pacote \textit{algorithm} é usado para escrever algoritmos em alto nível \cite{janos:2005:algpseudocode}. Já o pacote \textit{listings} serve para escrever os códigos em diversas linguagens de programação \cite{moses:2006:listings}.

Caso sejam utilizados os ambientes de algoritmos e código podem ser incluídos os comandos \comando{incluidelistaalgoritmos} e \comando{incluidelistacodigos} antes do \comando{begin\{document\}} para que a lista de algoritmos e a lista de código sejam criadas.


\section{Ambientes Matemáticos}

A classe \textit{icmc} provê os seguintes ambientes matemáticos:
\begin{itemize}
 \item Teoremas (\comando{begin\{teorema\}[\ ]} ... \comando{begin\{teorema\}});
 \item Proposição (\comando{begin\{proposicao\}[\ ]} ... \comando{begin\{proposicao\}});
 \item Lema (\comando{begin\{lema\}[\ ]} ... \comando{begin\{lema\}});
 \item Corolário (\comando{begin\{corolario\}[\ ]} ... \comando{begin\{corolario\}});
 \item Exemplo (\comando{begin\{exemplo\}[\ ]} ... \comando{begin\{exemplo\}});
 \item Observação (\comando{begin\{observacao\}[\ ]} ... \comando{begin\{observacao\}});
 \item Definição (\comando{begin\{definicao\}[\ ]} ... \comando{begin\{definicao\}});
 \item demonstracao (\comando{begin\{demonstracao\}[\ ]} ... \comando{begin\{demonstracao\}}).
\end{itemize}

Abaixo temos um exemplo de proposição com sua demonstração:
\begin{proposicao}
 Sejam $a$ e $b$ reais, tais que $0<a<b$. Então $a^2<b^2$.
\end{proposicao}
\begin{demonstracao}
 Pela hipótese concluímos que $(b+a)>0$ e $(b-a)>0$.

Como $b^2-a^2=(b+a)(b-a)$ concluímos que $b^2-a^2>0$, ou seja, $a^2<b^2$.
\end{demonstracao}

Neste documento tratamos brevemente apenas dos ambientes mencionados anteriormente. Contudo, para escrever expressões matemáticas complexas é preciso estudar uma documentação mais específica como em \citeonline{cassagojr:1997:amslatex}.

Alguns dos ambientes matemáticos da classe \textit{icmc} podem ser usados também para outras finalidades como exemplos e definições.


\section{Definição de outros ambientes}
\label{secao:outros-ambientes}

O classe \textit{icmc} permite a criação de outros ambientes, além dos citados nas seções anteriores, caso seja necessário. Isso é possível graças a extensão da classe \textit{abntex}. O \autoref{codigo:novo-ambiente} deve ser inserido antes do início do documento para criação de um ambiente para gráficos. Para definição de outros ambientes, basta seguir o modelo.


\begin{codigo}[caption={Definição do ambiente \textbf{grafico}}, label={codigo:novo-ambiente}, language=Tex, breaklines=true]
\makeatletter

% Novo list of (listings) para GRÁFICOS --------------------------
\newcommand{\graficoname}{Gráfico}
\newcommand{\graficorefname}{Gráfico}
\newcommand{\listofgraficosname}{Lista de gráficos}

\addto\captionsenglish{% ingles
    %% adjusts names from abnTeX2
    \newcommand{\graficoname}{Graph}
    \newcommand{\graficorefname}{Graph}
    \newcommand{\listofgraficosname}{List of graphs}
}

\newfloat[chapter]{grafico}{logr}{\graficoname}
\newlistof{listofgraficos}{logr}{\listgraficoname}
\newlistentry{grafico}{logr}{0}

% configurações para atender às regras da ABNT
\renewcommand{\thegrafico}{\thechapter.\@arabic\c@grafico}
\setfloatadjustment{grafico}{\centering}
\renewcommand{\cftgraficoname}{\graficoname\space}
\renewcommand*{\cftgraficoaftersnum}{\hfill\textendash\hfill}
% ----------------------------------------------------------------

\makeatother
\end{codigo}

A utilização do novo ambiente no texto segue conforme o \autoref{codigo:usar-novo-ambiente}.

\begin{codigo}[caption={Como usar o ambiente \textbf{grafico}}, label={codigo:usar-novo-ambiente}, language=Tex, breaklines=true]
\begin{grafico}[htb]
\caption{Caption do gráfico}
\label{gra:modelo}
Este é o conteúdo do gráfico.
\end{grafico}
\end{codigo}

Comandos como \comando{autoref\{gra:modelo\}} funcionam normalmente.

Para imprimir a "Lista de gráficos" no documento, insira o \autoref{codigo:lista-novo-ambiente} na classe \textit{icmc}, de modo que ele seja impresso após a "Lista de ilustrações". O código deve ser inserido após a linha 1244.


\begin{codigo}[caption={Código para inserir lista de gráficos}, label={codigo:lista-novo-ambiente}, language=Tex, breaklines=true]
% ---
% inserir lista de gráficos
% ---
\pdfbookmark[0]{\listofgraficosname}{logr}
\listofgraficos*
\cleardoublepage
% ---
\end{codigo}